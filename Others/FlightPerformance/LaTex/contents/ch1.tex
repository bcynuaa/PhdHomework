\section{作业要求及实现方式}

\subsection{作业要求}

在已知某飞行器飞行数据的情况下,计算其基本飞行性能,并对数据做可视化处理。

已知的飞行器飞行数据如下:

\begin{itemize}
    \item 飞机质量$m=263\text{kg}$;
    \item 不同迎角$\alpha$与飞行马赫数$Ma$下的升力系数$C_L$;
    \item 不同迎角$\alpha$与飞行马赫数$Ma$下的阻力系数系数$C_D$;
    \item 不同海拔高度$h$与飞行马赫数$Ma$下的加力推力大小$T_{AB}$;
    \item 不同海拔高度$h$与飞行马赫数$Ma$下的发动机推力大小$T$。
\end{itemize}

需要求取与绘制的飞行性能数据如下:

\begin{enumerate}
    \item 跨音速马赫数下的极曲线$C_D-C_L$;
    \item 不同迎角$\alpha$下,$C_{D0}(Ma)$、$A(Ma)$与$C_{l\alpha}(Ma)$的曲线图象;
    \item 加力与非加力状态下的推力曲线(即需用推力$T_R$的求取,$T_a$是已有数据);
    \item 加力与非加力状态下的理论飞行包线;
    \item 理论静升限的求取;
    \item SEP图像。
\end{enumerate}

\subsection{实现方式}

本次作业涉及数据的读取、处理、绘制,以及方程求解、插值等,因此本次作业选择的实现方式如下:

\begin{itemize}
    \item 编程语言选取julia,其兼具解释型语言的灵活与编译型语言的速度,适合做科学计算;
    \item 原始数据为XML格式,本次作业采用XMLDict.jl进行数据读取,并在程序内用struct管理飞行数据;
    \item 原始数据为离散点,在实际处理过程中需要插值处理,而简单的线性插值不能满足精度要求,因此本次作业采用Dierckx.jl包进行二次或三次的数据插值处理;
    \item 在计算过程中,会涉及到方程的求解问题,本次作业采用Roots.jl包求解非线性方程;
    \item 在计算SEP的$V_{v.\max}$时,需要进行极大值的求取,本次作业再用Optim.jl包进行处理;
    \item 因为涉及到函数拟合过程,自行推导并实现最小二乘拟合;
    \item 本次作业涉及到曲线绘图,采用Plots.jl包进行示意;部分图片采用tikz进行绘制;
    \item 为方便索引与公式书写,本次作业报告用LaTex撰写。
\end{itemize}